\documentclass[12pt]{article}
\usepackage[T2A]{fontenc}
\usepackage[utf8]{inputenc}
\usepackage{graphics}
\usepackage[english, russian]{babel}
\usepackage{csquotes}
\usepackage{graphicx}
\usepackage{amsmath}
\usepackage{longtable}
\usepackage[left=25mm, top=20mm, right=25mm, bottom=30mm,nohead,nofoot]{geometry}
\usepackage{verbatim}
\usepackage{hyperref}
\usepackage{amssymb,latexsym}  % Standard packages


\usepackage{MnSymbol}
\usepackage{mathrsfs}
\usepackage{amsthm}
\usepackage{indentfirst}
\usepackage[nottoc,numbib]{tocbibind}
\usepackage{float}

\usepackage{epigraph} %%% to make inspirational quotes.
\usepackage[all]{xy} %for XyPic'a
\usepackage{color} 
\usepackage{amscd} %для коммутативных диграмм
\usepackage{listings}

\setcounter{tocdepth}{4}
\graphicspath{ {./pic/} }

\addto\captionsrussian{\def\refname{6 \: Литература}}

\begin{document}

\begin{titlepage}
	\center
		Санкт-Петербургский Политехнический 
		университет Петра Великого
		Институт прикладной математики и механики
		\\ \textbf{Кафедра «Прикладная математика»}

	\vfill ~
	\textbf{
		\\ \large ЛАБОРАТОРНАЯ РАБОТА №2
		\\	\normalsize	
			Характеристики положения выборки
	}
	\\	по дисциплине 
	\\	"Математическая статистика"

	\vfill ~

	Выполнил студент гр. \textbf{33631/1} \\
	\textbf{Лансков.Н.В.} \\ 

\vfill

{\large}	Санкт-Петербург
\\ 2019
\end{titlepage}

%%%
% Table of conetnts 
%%%
% \settocdepth{chapter}
\tableofcontents
\newpage
% \listoffigures
% \newpage
\listoftables
\newpage
\pagebreak

% \setcounter{chapter}{1}

%%%
% Text
%%%
\section{Постановка задачи}
Любыми средствами сгенерировать выборки размеров $20,$ $60,$ $100$ элементов для $5$ти распределений \cite{distr_formulas}. Для каждой выборки вычислить $\overline{x},\; med\; x,\; Z_R,\; Z_Q,\; Z_{tr},$ при $r = \frac{n}{4}.$

Распределения:
\begin{enumerate}
\item Стандартное нормальное распределение
\item Стандартное распределение Коши
\item Распределение Лапласа с коэффициентом масштаба $\sqrt{2}$ и нулевым коэффициентом сдвига.
\item Равномерное распределение на отрезке $\left[-\sqrt{3}, \sqrt{3}\right]$
\item Распределение Пуассона со значением матожидания равным двум.
\end{enumerate}

\section{Теория}

\subsection{Выборочное среднее}

\begin{equation}
\overline{x} = \frac{1}{n}\sum_{i=1}^n x_i \hfill \label{eqn:average}
\end{equation}

\subsection{Выборочная медиана}

\begin{equation}
med\; x = \begin{cases}
x_{k+1}, & n = 2k+1\\
\frac{1}{2}\left(x_k+x_{k+1}\right), & n = 2k
\end{cases} \hfill  \label{eqn:med}
\end{equation}

\subsection{Полусумма экстремальных значений}

\begin{equation} 
Z_R = \frac{1}{2}\left(x_1+x_n\right) \hfill  \label{eqn:mean_extr}
\end{equation}

\subsection{Полусумма квартилей}

\begin{equation}
Z_Q = \frac{1}{2}\left(Z_{\frac{1}{4}}+Z_{\frac{3}{4}}\right) \hfill  
\label{eqn:quartiles}
\end{equation}

\subsection{Усечённое среднее}

\begin{equation}
Z_{tr} = \frac{1}{n - 2r}\sum_{i=r+1}^{n-r} x_i \hfill  \label{eqn:cut_mean}
\end{equation}

\pagebreak

\section{Реализация}
Выполнено средствами \textit{python} c применением библиотеки \textit{numpy}

\section{Результаты}

\begin{table}[H]
\caption{out1}
\label{tab:my_label1}
\begin{center}
\vspace{5mm}
\begin{tabular}{|c|}
\hline
-------------------------------------\\
\hline
normal\\
\hline
\end{tabular}
\end{center}
\end{table}

\begin{table}[H]
\caption{out2}
\label{tab:my_label2}
\begin{center}
\vspace{5mm}
\begin{tabular}{|c|}
\hline
n = 20    &average     &med         &Zr          &Zq          &Ztr r = n/4 \\
\hline
E =       &-0.013333   &0.009063    &-0.005522   &-0.012054   &-0.002018   \\
\hline
D =       &0.046945    &0.076766    &0.142835    &0.056061    &0.060165    \\
\hline
\end{tabular}
\end{center}
\end{table}

\begin{table}[H]
\caption{out3}
\label{tab:my_label3}
\begin{center}
\vspace{5mm}
\begin{tabular}{|c|c|c|c|c|c|}
\hline
n = 60    &average     &med         &Zr          &Zq          &Ztr r = n/4 \\
\hline
E =       &0.001963    &-0.003607   &-0.010240   &-0.004640   &-0.001696   \\
\hline
D =       &0.016037    &0.025284    &0.107303    &0.020871    &0.020296    \\
\hline
\end{tabular}
\end{center}
\end{table}

\begin{table}[H]
\caption{out4}
\label{tab:my_label4}
\begin{center}
\vspace{5mm}
\begin{tabular}{|c|c|c|c|c|c|}
\hline
n = 100   &average     &med         &Zr          &Zq          &Ztr r = n/4 \\
\hline
E =       &-0.006258   &0.000041    &0.004206    &-0.000581   &-0.000962   \\
\hline
D =       &0.009937    &0.015242    &0.088866    &0.012238    &0.012099    \\
\hline
------------------------------------- & & & & &
\\
\hline
cauchy & & & & &
\\
\hline
\end{tabular}
\end{center}
\end{table}

\begin{table}[H]
\caption{out5}
\label{tab:my_label5}
\begin{center}
\vspace{5mm}
\begin{tabular}{|c|c|c|c|c|c|}
\hline
n = 20    &average     &med         &Zr          &Zq          &Ztr r = n/4 \\
\hline
E =       &0.899701    &-0.009714   &-4.927604   &0.013856    &-0.008149   \\
\hline
D =       &38271.053130&0.142732    &30015.253592&0.307986    &0.152662    \\
\hline
\end{tabular}
\end{center}
\end{table}

\begin{table}[H]
\caption{out6}
\label{tab:my_label6}
\begin{center}
\vspace{5mm}
\begin{tabular}{|c|c|c|c|c|c|}
\hline
n = 60    &average     &med         &Zr          &Zq          &Ztr r = n/4 \\
\hline
E =       &-0.844512   &-0.004553   &-6.426374   &0.001648    &0.007870    \\
\hline
D =       &1663.090280 &0.042251    &550349.336159&0.093719    &0.042292    \\
\hline
\end{tabular}
\end{center}
\end{table}

\begin{table}[H]
\caption{out7}
\label{tab:my_label7}
\begin{center}
\vspace{5mm}
\begin{tabular}{|c|c|c|c|c|c|}
\hline
n = 100   &average     &med         &Zr          &Zq          &Ztr r = n/4 \\
\hline
E =       &0.475302    &0.001242    &18.325690   &-0.004266   &0.002305    \\
\hline
D =       &178.376488  &0.026131    &1996325.674308&0.052982    &0.024151    \\
\hline
------------------------------------- & & & & &
\\
\hline
laplace & & & & &
\\
\hline
\end{tabular}
\end{center}
\end{table}

\begin{table}[H]
\caption{out8}
\label{tab:my_label8}
\begin{center}
\vspace{5mm}
\begin{tabular}{|c|c|c|c|c|c|}
\hline
n = 20    &average     &med         &Zr          &Zq          &Ztr r = n/4 \\
\hline
E =       &0.001431    &0.006146    &0.000307    &-0.001240   &-0.003024   \\
\hline
D =       &0.053531    &0.033672    &0.424217    &0.050205    &0.030953    \\
\hline
\end{tabular}
\end{center}
\end{table}

\begin{table}[H]
\caption{out9}
\label{tab:my_label9}
\begin{center}
\vspace{5mm}
\begin{tabular}{|c|c|c|c|c|c|}
\hline
n = 60    &average     &med         &Zr          &Zq          &Ztr r = n/4 \\
\hline
E =       &-0.008891   &-0.003148   &0.019806    &-0.002361   &0.003333    \\
\hline
D =       &0.015753    &0.009502    &0.407433    &0.015607    &0.010857    \\
\hline
\end{tabular}
\end{center}
\end{table}

\begin{table}[H]
\caption{out10}
\label{tab:my_label10}
\begin{center}
\vspace{5mm}
\begin{tabular}{|c|c|c|c|c|c|}
\hline
n = 100   &average     &med         &Zr          &Zq          &Ztr r = n/4 \\
\hline
E =       &-0.001476   &-0.001714   &0.017273    &0.001879    &0.001884    \\
\hline
D =       &0.009809    &0.005924    &0.428472    &0.010128    &0.006141    \\
\hline
------------------------------------- & & & & &
\\
\hline
uniform & & & & &
\\
\hline
\end{tabular}
\end{center}
\end{table}

\begin{table}[H]
\caption{out11}
\label{tab:my_label11}
\begin{center}
\vspace{5mm}
\begin{tabular}{|c|c|c|c|c|c|}
\hline
n = 20    &average     &med         &Zr          &Zq          &Ztr r = n/4 \\
\hline
E =       &0.002672    &-0.010210   &-0.000425   &0.016886    &-0.006024   \\
\hline
D =       &0.048714    &0.136940    &0.012372    &0.068931    &0.095201    \\
\hline
\end{tabular}
\end{center}
\end{table}

\begin{table}[H]
\caption{out12}
\label{tab:my_label12}
\begin{center}
\vspace{5mm}
\begin{tabular}{|c|c|c|c|c|c|}
\hline
n = 60    &average     &med         &Zr          &Zq          &Ztr r = n/4 \\
\hline
E =       &-0.004400   &0.010762    &0.000296    &0.003209    &-0.005259   \\
\hline
D =       &0.016255    &0.048588    &0.001512    &0.023774    &0.034007    \\
\hline
\end{tabular}
\end{center}
\end{table}

\begin{table}[H]
\caption{out13}
\label{tab:my_label13}
\begin{center}
\vspace{5mm}
\begin{tabular}{|c|c|c|c|c|c|}
\hline
n = 100   &average     &med         &Zr          &Zq          &Ztr r = n/4 \\
\hline
E =       &0.004468    &0.000156    &0.001146    &-0.001466   &-0.007706   \\
\hline
D =       &0.010249    &0.030074    &0.000581    &0.014292    &0.018242    \\
\hline
------------------------------------- & & & & &
\\
\hline
poisson & & & & &
\\
\hline
\end{tabular}
\end{center}
\end{table}

\begin{table}[H]
\caption{out14}
\label{tab:my_label14}
\begin{center}
\vspace{5mm}
\begin{tabular}{|c|c|c|c|c|c|}
\hline
n = 20    &average     &med         &Zr          &Zq          &Ztr r = n/4 \\
\hline
E =       &2.008650    &1.870000    &2.509500    &1.920375    &1.860000    \\
\hline
D =       &0.102498    &0.182600    &0.275160    &0.137644    &0.127640    \\
\hline
\end{tabular}
\end{center}
\end{table}

\begin{table}[H]
\caption{out15}
\label{tab:my_label15}
\begin{center}
\vspace{5mm}
\begin{tabular}{|c|c|c|c|c|c|}
\hline
n = 60    &average     &med         &Zr          &Zq          &Ztr r = n/4 \\
\hline
E =       &1.988967    &1.927500    &2.968500    &1.939500    &1.853567    \\
\hline
D =       &0.034121    &0.061994    &0.245258    &0.032590    &0.044723    \\
\hline
\end{tabular}
\end{center}
\end{table}

\begin{table}[H]
\caption{out16}
\label{tab:my_label16}
\begin{center}
\vspace{5mm}
\begin{tabular}{|c|c|c|c|c|c|}
\hline
n = 100   &average     &med         &Zr          &Zq          &Ztr r = n/4 \\
\hline
E =       &1.999050    &1.972500    &3.154000    &1.973500    &1.838020    \\
\hline
D =       &0.020132    &0.024494    &0.244784    &0.012735    &0.028104    \\
\hline
\end{tabular}
\end{center}
\end{table}

\section{Обсуждение}

\addcontentsline{toc}{section}{6 \: Литература} 
\begin{thebibliography}{}
    \bibitem{wiki} \url{https://www.wikipedia.org/}
\end{thebibliography}

\section{Выводы}
В результате работы были построены графики для трёх выборок разных мощностей для каждого из рассматриваемых распределений. Из графиков видно, что с увеличением мощности выборки, диаграмма всё менее отклоняется от теоретического значения. Это иллюстрирует факт того, что при стремлении можности выборки к бесконечности диаграмма выборки будет оцениваться теоретической кривой с любой интересующей нас точностью.
\par
Конечно, за счёт того что размеры выборок довольно малы, то могут наблюдаться некоторые "выбросы" в конкретных точках (особенно заметно на самых левых графиках для мощности 10). Это объясняется тем, что значения выборки генерируются случайным образом и данных на такой мощности для построения теоретических оценок оказывается недостаточно.

\section{Приложения}

Исходники: \url{https://github.com/LanskovNV/math_statistics}

\end{document}

