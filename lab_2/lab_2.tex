\documentclass[12pt]{article}
\usepackage[T2A]{fontenc}
\usepackage[utf8]{inputenc}
\usepackage{graphics}
\usepackage[english, russian]{babel}
\usepackage{csquotes}
\usepackage{graphicx}
\usepackage{amsmath}
\usepackage{longtable}
\usepackage[left=25mm, top=20mm, right=25mm, bottom=30mm,nohead,nofoot]{geometry}
\usepackage{verbatim}
\usepackage{hyperref}
\usepackage{amssymb,latexsym}  % Standard packages
\usepackage{MnSymbol}
\usepackage{mathrsfs}
\usepackage{amsthm}
\usepackage{indentfirst}
\usepackage[nottoc,numbib]{tocbibind}
\usepackage{float}

%******************************************************************
\usepackage{mathrsfs}
\usepackage{fancyhdr}
\pagestyle{fancy}
\lhead{}
\chead{}
\rhead{}
\lfoot{}
\cfoot{} 
\rfoot{\normalsize\thepage}
\renewcommand{\headrulewidth}{0pt}
\renewcommand{\footrulewidth}{0pt}
\newcommand{\RomanNumeralCaps}[1]
    {\MakeUppercase{\romannumeral #1}}
\usepackage{amssymb,latexsym}  % Standard packages
\usepackage[utf8]{inputenc}
\usepackage[russian]{babel}
\usepackage{MnSymbol}
\usepackage{mathrsfs}
\usepackage{amsmath,amsthm}
\usepackage{indentfirst}
\usepackage{graphicx}%,vmargin}
\usepackage{graphicx}
\graphicspath{{pictures/}} 
\usepackage{verbatim}
\usepackage{color}
\usepackage[nottoc,numbib]{tocbibind}
\usepackage{float}
\usepackage{listings}
\definecolor{codegreen}{rgb}{0,0.6,0}
\definecolor{codegray}{rgb}{0.5,0.5,0.5}
\definecolor{codepurple}{rgb}{0.58,0,0.82}
\definecolor{backcolour}{rgb}{0.95,0.95,0.92}
\lstdefinestyle{mystyle}{
    backgroundcolor=\color{backcolour},   
    commentstyle=\color{codegreen},
    keywordstyle=\color{magenta},
    numberstyle=\tiny\color{codegray},
    stringstyle=\color{codepurple},
    basicstyle=\footnotesize,
    breakatwhitespace=false,         
    breaklines=true,                 
    captionpos=b,                    
    keepspaces=true,                 
    numbers=left,                    
    numbersep=5pt,                  
    showspaces=false,                
    showstringspaces=false,
    showtabs=false,                  
    tabsize=2
}
\lstset{style=mystyle}
\usepackage{url}
\urldef\myurl\url{foo%.com}
\DeclareGraphicsExtensions{.pdf,.png,.jpg}% -- настройка картинок
\usepackage{epigraph} %%% to make inspirational quotes.
\usepackage[all]{xy} %for XyPic'a
\usepackage{color} 
\usepackage{amscd} %для коммутативных диграмм
%******************************************************************

\setcounter{tocdepth}{4}
\graphicspath{ {./pic/} }

\addto\captionsrussian{\def\refname{6 \: Литература}}

\begin{document}

\begin{titlepage}
	\center
		Санкт-Петербургский Политехнический 
		университет Петра Великого
		Институт прикладной математики и механики
		\\ \textbf{Кафедра «Прикладная математика»}

	\vfill ~
	\textbf{
		\\ \large ЛАБОРАТОРНАЯ РАБОТА №2
		\\	\normalsize	
			Характеристики положения выборки
	}
	\\	по дисциплине 
	\\	"Математическая статистика"

	\vfill ~

	Выполнил студент гр. \textbf{33631/1} \\
	\textbf{Лансков.Н.В.} \\ 

\vfill

{\large}	Санкт-Петербург
\\ 2019
\end{titlepage}

%%%
% Table of conetnts 
%%%
% \settocdepth{chapter}
\tableofcontents
\newpage
% \listoffigures
% \newpage
\listoftables
\newpage
\pagebreak

% \setcounter{chapter}{1}

%%%
% Text
%%%
\section{Постановка задачи}
Любыми средствами сгенерировать выборки размеров $20,$ $60,$ $100$ элементов для $5$ти распределений \cite{distr_formulas}. Для каждой выборки вычислить $\overline{x},\; med\; x,\; Z_R,\; Z_Q,\; Z_{tr},$ при $r = \frac{n}{4}.$

Распределения:
\begin{enumerate}
\item Стандартное нормальное распределение
\item Стандартное распределение Коши
\item Распределение Лапласа с коэффициентом масштаба $\sqrt{2}$ и нулевым коэффициентом сдвига.
\item Равномерное распределение на отрезке $\left[-\sqrt{3}, \sqrt{3}\right]$
\item Распределение Пуассона со значением матожидания равным двум.
\end{enumerate}

\section{Теория}

\subsection{Выборочное среднее}

\begin{equation}
\overline{x} = \frac{1}{n}\sum_{i=1}^n x_i \hfill \label{eqn:average}
\end{equation}

\subsection{Выборочная медиана}

\begin{equation}
med\; x = \begin{cases}
x_{k+1}, & n = 2k+1\\
\frac{1}{2}\left(x_k+x_{k+1}\right), & n = 2k
\end{cases} \hfill  \label{eqn:med}
\end{equation}

\subsection{Полусумма экстремальных значений}

\begin{equation} 
Z_R = \frac{1}{2}\left(x_1+x_n\right) \hfill  \label{eqn:mean_extr}
\end{equation}

\subsection{Полусумма квартилей}

\begin{equation}
Z_Q = \frac{1}{2}\left(Z_{\frac{1}{4}}+Z_{\frac{3}{4}}\right) \hfill  
\label{eqn:quartiles}
\end{equation}

\subsection{Усечённое среднее}

\begin{equation}
Z_{tr} = \frac{1}{n - 2r}\sum_{i=r+1}^{n-r} x_i \hfill  \label{eqn:cut_mean}
\end{equation}

\pagebreak

\section{Реализация}
Выполнено средствами \textit{python} c применением библиотеки \textit{numpy}

\section{Результаты}

\begin{table}[H]
\caption{normal}
\label{tab:my_label1}
\begin{center}
\vspace{5mm}
\begin{tabular}{|c|c|c|c|c|c|}
\hline
n = 20    &average     &med         &Zr          &Zq          &Ztr r = n/4 \\
\hline
E =       &0.003889    &-0.001204   &0.004247    &-0.002884   &0.014986    \\
\hline
D =       &0.050065    &0.069428    &0.135281    &0.056127    &0.060241    \\
\hline
n = 60    &average     &med         &Zr          &Zq          &Ztr r = n/4 \\
\hline
E =       &0.005639    &-0.001905   &0.006317    &0.003647    &-0.005183   \\
\hline
D =       &0.017071    &0.024751    &0.109666    &0.020065    &0.019645    \\
\hline
n = 100   &average     &med         &Zr          &Zq          &Ztr r = n/4 \\
\hline
E =       &-0.000990   &-0.008915   &-0.003211   &-0.007140   &0.000411    \\
\hline
D =       &0.010270    &0.015038    &0.091469    &0.012224    &0.011148    \\
\hline
\end{tabular}
\end{center}
\end{table}

\begin{table}[H]
\caption{cauchy}
\label{tab:my_label2}
\begin{center}
\vspace{5mm}
\begin{tabular}{|c|c|c|c|c|c|}
\hline
n = 20    &average     &med         &Zr          &Zq          &Ztr r = n/4 \\
\hline
E =       &0.574031    &-0.009207   &4.314604    &0.009490    &0.024569    \\
\hline
D =       &380.673006  &0.127344    &24864.038520&0.290641    &0.156087    \\
\hline
n = 60    &average     &med         &Zr          &Zq          &Ztr r = n/4 \\
\hline
E =       &16.889466   &0.007131    &-14.877379  &-0.006227   &-0.007330   \\
\hline
D =       &218415.033090&0.039049    &268917.253813&0.087454    &0.042852    \\
\hline
n = 100   &average     &med         &Zr          &Zq          &Ztr r = n/4 \\
\hline
E =       &-1.147046   &-0.000622   &-23.949387  &-0.002369   &-0.008024   \\
\hline
D =       &604.646590  &0.024850    &1995730.222185&0.051894    &0.023969    \\
\hline
\end{tabular}
\end{center}
\end{table}

\begin{table}[H]
\caption{laplace}
\label{tab:my_label3}
\begin{center}
\vspace{5mm}
\begin{tabular}{|c|c|c|c|c|c|}
\hline
n = 20    &average     &med         &Zr          &Zq          &Ztr r = n/4 \\
\hline
E =       &-0.001627   &-0.000952   &0.032669    &0.003535    &0.005730    \\
\hline
D =       &0.045803    &0.031761    &0.402496    &0.046401    &0.033325    \\
\hline
n = 60    &average     &med         &Zr          &Zq          &Ztr r = n/4 \\
\hline
E =       &-0.006856   &-0.002807   &0.050344    &-0.000216   &-0.000712   \\
\hline
D =       &0.016743    &0.009838    &0.431677    &0.017332    &0.009962    \\
\hline
n = 100   &average     &med         &Zr          &Zq          &Ztr r = n/4 \\
\hline
E =       &-0.000800   &0.001184    &-0.024119   &-0.002611   &-0.000909   \\
\hline
D =       &0.009861    &0.005529    &0.409545    &0.009740    &0.006101    \\
\hline
\end{tabular}
\end{center}
\end{table}

\begin{table}[H]
\caption{uniform}
\label{tab:my_label4}
\begin{center}
\vspace{5mm}
\begin{tabular}{|c|c|c|c|c|c|}
\hline
n = 20    &average     &med         &Zr          &Zq          &Ztr r = n/4 \\
\hline
E =       &-0.004788   &0.011925    &-0.000665   &-0.003460   &-0.002870   \\
\hline
D =       &0.049107    &0.134658    &0.013457    &0.071206    &0.097207    \\
\hline
n = 60    &average     &med         &Zr          &Zq          &Ztr r = n/4 \\
\hline
E =       &0.001583    &-0.004005   &-0.001657   &-0.005769   &-0.003667   \\
\hline
D =       &0.016670    &0.045087    &0.001706    &0.024632    &0.033966    \\
\hline
n = 100   &average     &med         &Zr          &Zq          &Ztr r = n/4 \\
\hline
E =       &0.000676    &0.004193    &-0.000025   &-0.005675   &0.004133    \\
\hline
D =       &0.010255    &0.028780    &0.000621    &0.015457    &0.019007    \\
\hline
\end{tabular}
\end{center}
\end{table}

\begin{table}[H]
\caption{poisson}
\label{tab:my_label5}
\begin{center}
\vspace{5mm}
\begin{tabular}{|c|c|c|c|c|c|}
\hline
n = 20    &average     &med         &Zr          &Zq          &Ztr r = n/4 \\
\hline
E =       &2.016900    &1.868000    &2.531500    &1.899250    &1.865800    \\
\hline
D =       &0.099684    &0.179576    &0.290758    &0.133599    &0.115410    \\
\hline
n = 60    &average     &med         &Zr          &Zq          &Ztr r = n/4 \\
\hline
E =       &2.005733    &1.932500    &2.945000    &1.936875    &1.843400    \\
\hline
D =       &0.033034    &0.054194    &0.233475    &0.034812    &0.041836    \\
\hline
n = 100   &average     &med         &Zr          &Zq          &Ztr r = n/4 \\
\hline
E =       &1.997070    &1.961500    &3.128000    &1.963625    &1.844780    \\
\hline
D =       &0.020288    &0.033268    &0.217616    &0.017067    &0.027989    \\
\hline
\end{tabular}
\end{center}
\end{table}

\section{Обсуждение}

\addcontentsline{toc}{section}{6 \: Литература} 
\begin{thebibliography}{}
    \bibitem{wiki} \url{https://www.wikipedia.org/}
\end{thebibliography}

\section{Выводы}
В результате работы были построены графики для трёх выборок разных мощностей для каждого из рассматриваемых распределений. Из графиков видно, что с увеличением мощности выборки, диаграмма всё менее отклоняется от теоретического значения. Это иллюстрирует факт того, что при стремлении можности выборки к бесконечности диаграмма выборки будет оцениваться теоретической кривой с любой интересующей нас точностью.
\par
Конечно, за счёт того что размеры выборок довольно малы, то могут наблюдаться некоторые "выбросы" в конкретных точках (особенно заметно на самых левых графиках для мощности 10). Это объясняется тем, что значения выборки генерируются случайным образом и данных на такой мощности для построения теоретических оценок оказывается недостаточно.

\section{Приложения}

Исходники: \url{https://github.com/LanskovNV/math_statistics}

\end{document}

